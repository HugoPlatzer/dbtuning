\documentclass[11pt]{scrartcl}

\title{
  \textbf{\large Database Tuning -- Assignment 3}\\
  Index Tuning
}

\author{
 Group Name (e.g. A1, B5, B3)\\
 \large Lastname1 Firstname1, StudentID1 \\
 \large Lastname2 Firstname2, StudentID2 \\
 \large Lastname3 Firstname3, StudentID3 
}

\begin{document}

\maketitle

\medskip

\noindent\textbf{Database system and version:} {\it e.g., Oracle 12c}

\subsection*{Question 1: Index Data Structures} Which index data structures (e.g., $B^+$-tree
index) are supported?

\smallskip

B+-Tree\\
Hash\\
GIN (Generalized Inverted Indexes)\\
GiST (Generalized Search Tree)\\
SP-GiST (Space partitioned GiST)\\
BRIN (Block range indexes)

\subsection*{Question 2: Clustered Indexes} Discuss how the system
supports clustered indexes, in particular:

\paragraph{2a)} How do you create a clustered index on {\tt ssnum}?
Show the query.\footnote{Give the queries for creating a hash index
  \emph{and} a $B^+$-tree index if both of them are supported.}

\smallskip
Since PostgreSQL is using heap tables, there doesn't exist a clustered index. We can however order the index using the cluster method.\\
First we need to create an index. After that we can cluster our table using that index.\\
Tuples that are inserted after this, however are not ordered.

{\small
\begin{verbatim}
B+-Tree: CREATE INDEX ssnum_index ON Employee (ssnum);
Hash: CREATE INDEX ssnum_index ON Employee USING hash (ssnum);

CLUSTER Employee USING ssnum_index;
\end{verbatim}
}

\paragraph{2b)} Are clustered indexes on non-key attributes supported, e.g.,
on {\tt name}?  Show the query.

\smallskip

No. CLUSTER instructs PostgreSQL to cluster the table specified by table_name based on the index specified by index_name. The index must already have been defined on table_name.

{\small
\begin{verbatim}
SQL QUERY ...
\end{verbatim}
}

\paragraph{2c)} Is the clustered index dense or sparse?


\smallskip

Dense, since newly inserted tuples are not ordered.

\paragraph{2d)} How does the system deal with overflows in clustered indexes?
How is the fill factor controlled?

\smallskip

Your answer\dots

\paragraph{2e)} Discuss any further characteristics of the system
related to clustered indexes that are relevant to a database
tuner?

\smallskip

Clustering only sorts once. Further inserted tuples are not ordered.

\subsection*{Question 3: Non-Clustered Indexes}

Discuss how the system supports non-clustered indexes, in
particular:

\paragraph{3a)} How do you create a non-clustered index on {\tt
  (dept,salary)}? Show the query.$^1$

\smallskip

Just create an index on (dept,salary), PostgreSQL doesn't have clustered indices.

{\small
\begin{verbatim}
CREATE INDEX dept_salary_index ON Employee (dept,salary);
\end{verbatim}
}

\paragraph{3b)} Can the system take advantage of covering indexes? What if the
index covers the query, but the condition is not a prefix of the
attribute sequence {\tt (dept,salary)}?


\smallskip

Your answer\dots

\paragraph{3c)} Discuss any further characteristics of the system related to
non-clustered indexes that are relevant to a database tuner?

\smallskip

Your answer\dots

\subsection*{Question 4: Key Compression and Page Size} If your system
supports $B^+$-trees, what kind of key compression (if any) does it
support?  How large is the default disk page? Can it be changed?


\smallskip

Your answer\dots


\bigskip

\noindent Time in hours per person: {\bf XXX}

\bigskip

\begin{center}
  \begin{tabular}{c}
    \hline
    {\bf Important:} Reference your information sources!
    \\\hline
  \end{tabular}
\end{center}

\end{document}
