\documentclass[11pt]{scrartcl}
\usepackage{url}
\setlength{\parindent}{0pt}
\title{
  \textbf{\large Database Tuning -- Assignment 3}\\
  Index Tuning
}

\author{
 Group Name A3\\
 \large Platzer Hugo, 1421579 \\
 \large Strohmeier Mario, 1422959
}

\begin{document}

\maketitle

\medskip

\noindent\textbf{Database system and version:} {\it PostgreSQL 9.4.9}

\subsection*{Question 1: Index Data Structures} Which index data structures (e.g., $B^+$-tree
index) are supported?
\footnote{Postgres documentation on index types \url{https://www.postgresql.org/docs/9.5/static/indexes-types.html}}

\vspace{1em}
B+-Tree\\
Hash\\
GIN (Generalized Inverted Indexes)\\
GiST (Generalized Search Tree)\\
SP-GiST (Space partitioned GiST)\\
BRIN (Block range indexes)

\subsection*{Question 2: Clustered Indexes} Discuss how the system
supports clustered indexes, in particular:

\paragraph{2a)} How do you create a clustered index on {\tt ssnum}?
Show the query.
\footnote{\label{pdoc:cluster} Postgres documentation on CLUSTER command \url{https://www.postgresql.org/docs/9.5/static/sql-cluster.html}}

\vspace{1em}
When marking columns as primary key, Postgres automatically creates
a $B^+$-tree index for them; otherwise it could not check the uniqueness
efficiently.

{\small
\begin{verbatim}
B+-Tree: ALTER TABLE Employee ADD PRIMARY KEY (ssnum);
CLUSTER Employee USING ssnum_pkey;

Hash: CREATE INDEX ssnum_index ON Employee USING hash (ssnum);
CLUSTER Employee USING ssnum_index;
\end{verbatim}
}

\paragraph{2b)} Are clustered indexes on non-key attributes supported, e.g.,
on {\tt name}?  Show the query.

\vspace{1em}
This works in a similar fashion: You first create a regular index
on the attribute, then cluster the data by the index. \textsuperscript{\ref{pdoc:cluster}}

{\small
\begin{verbatim}
CREATE INDEX name_index ON Employee (name);
CLUSTER Employee USING name_index;
\end{verbatim}
}

\paragraph{2c)} Is the clustered index dense or sparse?

\vspace{1em}
Since Postgres only supports heap
tables (unordered tuples on disk), a dense index is required to address
every tuple. \footnote{http://stackoverflow.com/questions/19987786/postgresql-indices-are-they-dense-or-sparse}

\paragraph{2d)} How does the system deal with overflows in clustered indexes?
How is the fill factor controlled?

\vspace{1em}
When creating an index, Postgres allows one to specify the fillfactor like so:

{\small
\begin{verbatim}
CREATE INDEX name_index ON Employee (name) WITH (fillfactor=70);
\end{verbatim}
}

This means the leaf nodes of the $B^+$-tree will only be 70\% full
when the index is created. This allows new records to be added for some time
until index pages need to be split. It has the drawback of making the index
larger than necessary and thus degrading read performance.
\footnote{Postgres documentation on creating indices https://www.postgresql.org/docs/9.3/static/sql-createindex.html}

\paragraph{2e)} Discuss any further characteristics of the system
related to clustered indexes that are relevant to a database
tuner?

\smallskip

\vspace{1em}
Since PostgreSQL only supports heap tables, a clustered index in the strict sense
is not possible. We can however cluster the data using a given index. The data
gets physically sorted on the disk, clustering however is a one-time operation
and inserted/deleted data is not automatically clustered, leading to performance
degradation.
\footnote{\label{so:cluster}http://stackoverflow.com/questions/27978157/cluster-and-non-cluster-index-in-postgresql}

\subsection*{Question 3: Non-Clustered Indexes}

Discuss how the system supports non-clustered indexes, in
particular:

\paragraph{3a)} How do you create a non-clustered index on {\tt
  (dept,salary)}? Show the query.

\vspace{1em}
Just create an index on (dept,salary), Postgres doesn't have clustered indices.
\textsuperscript{\ref{so:cluster}}

Postgres does not support multi-column hash indices.
\footnote{Gives an error trying to create the index.}

{\small
\begin{verbatim}
B+-Tree: CREATE INDEX dept_salary_index ON Employee (dept,salary);
\end{verbatim}
}

\paragraph{3b)} Can the system take advantage of covering indexes? What if the
index covers the query, but the condition is not a prefix of the
attribute sequence {\tt (dept,salary)}?

\vspace{1em}
Postgres supports covering indexes:
\footnote{Postgres documentation on index-only scans
\url{https://wiki.postgresql.org/wiki/Index-only_scans}}

{\small
\begin{verbatim}
Aggregate  (cost=1862.31..1862.32 rows=1 width=2)
  ->  Index Only Scan using dept_salary_index on employee  (cost=0.29..1862.31 rows=2 width=2)
        Index Cond: (salary = 1000)
\end{verbatim}
}

If the condition is not a prefix, Postgres uses a bitmap heap scan.

{\small
\begin{verbatim} 
Aggregate  (cost=966.63..966.64 rows=1 width=4)
  ->  Bitmap Heap Scan on employee  (cost=74.39..956.92 rows=3883 width=4)
        Recheck Cond: ((dept)::text = 'A'::text)
        ->  Bitmap Index Scan on dept_salary_index  (cost=0.00..73.42 rows=3883 width=0)
              Index Cond: ((dept)::text = 'A'::text)
\end{verbatim}
}

\paragraph{3c)} Discuss any further characteristics of the system related to
non-clustered indexes that are relevant to a database tuner?

\smallskip

Your answer\dots

\subsection*{Question 4: Key Compression and Page Size} If your system
supports $B^+$-trees, what kind of key compression (if any) does it
support?  How large is the default disk page? Can it be changed?


\vspace{1em}
Postgres does not support index key compression.
\footnote{Since implementing this was given as homework to students,
we conclude Postgres does not support the feature.
\url{http://inst.eecs.berkeley.edu/~cs186/fa06/hw/hw1/}}

The default disk page size is 8 kB. This can be changed when compiling
Postgres.
\footnote{Postgres documentation on disk page layout 
https://www.postgresql.org/docs/9.5/static/storage-page-layout.html}

\bigskip

\noindent Time in hours per person: {\bf 3.5}

\bigskip

\begin{center}
  \begin{tabular}{c}
    \hline
    {\bf Important:} Reference your information sources!
    \\\hline
  \end{tabular}
\end{center}

\end{document}
