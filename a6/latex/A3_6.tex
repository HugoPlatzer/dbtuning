\documentclass[11pt]{scrartcl}

\usepackage[top=2cm]{geometry}
\usepackage{url,hyperref}

\title{
  \textbf{\large Database Tuning -- Assignment 6}\\
  Concurrency Tuning
}

\author{
 Group Name A3\\
 \large Platzer Hugo, 1421579 \\
 \large Strohmeier Mario, 1422959
}

\begin{document}

\maketitle

\section{Description of Setup}
All queries were run from the computer room of the facility on the 'biber' server.

Before each run, the Accounts (account, balance) table is deleted and refilled
by a separate program with the appropriate values.
The ExecutorService class is used to run Transaction threads in parallel, it 
also allows to set a limit for maximum concurrent threads.

Every Transaction thread first creates its own Connection, disables auto-commit (this allows
to build a transaction that consists of multiple statements / queries
, sets the isolation level,
then tries to execute all the statements in the transaction and commit. If this fails, i.e. a SQLException
is generated, it is assumed that the query failed because of serialization problems. Then a rollback is issued
and the transaction tried again from the first statment. This is repeated until the transaction commits.

The main thread waits until all Transaction threads finished execution. Time is measured between
starting the first thread and all of them having completed.

\section*{Task 1}

\subsection*{Read Committed}

Throughput and correctness for solution (a) with serialization level
{\tt\small READ COMMITTED}.

\bigskip

\begin{tabular}{c|c|c}
  \#Concurrent Transactions & Throughput [transactions/sec] & Correctness
  \\\hline
  1 & & \\
  2 & & \\
  3 & & \\
  4 & & \\
  5 & & \\    
\end{tabular}

\medskip

\subsection*{Serializable}

Throughput and correctness for solution (a) with serialization level
{\tt\small SERIALIZABLE}.

\bigskip

\begin{tabular}{c|c|c}
  \#Concurrent Transactions & Throughput [transactions/sec] & Correctness
  \\\hline
  1 & & \\
  2 & & \\
  3 & & \\
  4 & & \\
  5 & & \\    
\end{tabular}

\medskip

\section*{Task 2}

\subsection*{Read Committed}

Throughput and correctness for solution (b) with serialization level
{\tt\small READ COMMITTED}.

\bigskip

\begin{tabular}{c|c|c}
  \#Concurrent Transactions & Throughput [transactions/sec] & Correctness
  \\\hline
  1 & & \\
  2 & & \\
  3 & & \\
  4 & & \\
  5 & & \\    
\end{tabular}

\medskip

\subsection*{Serializable}

Throughput and correctness for solution (b) with serialization level
{\tt\small SERIALIZABLE}.

\bigskip

\begin{tabular}{c|c|c}
  \#Concurrent Transactions & Throughput [transactions/sec] & Correctness
  \\\hline
  1 & & \\
  2 & & \\
  3 & & \\
  4 & & \\
  5 & & \\    
\end{tabular}

\medskip

\section*{Task 3: Discussion}

Discuss the outcome and explain the difference between the isolation
levels in PostgreSQL with respect to your experiment.

Explain {\bf with your own words} how PostgreSQL deals with updates in
the different isolation levels, within a transaction and within a
single SQL command. Explicitly explain why you got the experimental
results of Task~1 and Task~2.

\subsection*{Task 1}

Discuss outcome of task 1 here.

\subsection*{Task 2}

Discuss outcome of task 2 here.

\bigskip

\noindent Time in hours per person: {\bf 4}

\bigskip

\begin{center}
  \begin{tabular}{c}
    \hline
    {\bf Important:} Reference your information sources!
    \\\hline
  \end{tabular}
\end{center}

\end{document}
