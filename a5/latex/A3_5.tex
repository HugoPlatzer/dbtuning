\documentclass[11pt]{scrartcl}

\usepackage[top=2cm]{geometry}
%\pagestyle{empty}

\title{
  \textbf{\large Database Tuning -- Assignment 5}\\
  Join Tuning
}

\author{
 Group Name A3\\
 \large Platzer Hugo, 1421579 \\
 \large Strohmeier Mario, 1422959
}

\begin{document}

\maketitle

\section{Description of Setup}
All queries were run from the computer room of the facility on the 'biber' server.

Data was inserted into the Postgres database using the \textbackslash copy command from the psql prompt as in
Assignment 1.

Time was measured using the Postgres "\textbackslash timing on" command

We deleted all tuples and reinserted them when executing a query that should be run
without clustering.

\section{Join Strategies Proposed by System}

\paragraph{Join strategies}

\begin{flushleft}
\begin{tabular}{l|l|l}
  Indexes & Join Strategy Q1 & Join Strategy Q2\\
  \hline
  no index & Hash Join & Hash Join  \\
  unique non-clustering on {\tt Publ.pubID} & Hash Join  & Nested Loop Join \\
  clustering on {\tt Publ.pubID} and {\tt Auth.pubID} & Hash Join & Nested Loop Join \\
\end{tabular}
\end{flushleft}

\paragraph{Discussion}
It is not surprising that Hash Join is chosen when there are no indices: Nested Loop Join
is extremely slow when there are no indices (needs to do sequential scan), Merge Join would
require sorting both relations, which we assume to be slower than creating a hashtable on pubid
for the smaller relation (publ). This is supported by looking at the clustering(sorting) times
(ca. 80s for auth and 25s for publ), much longer than the join queries' execution time.

When an index exists on the join attribute, an index nested loop join becomes possible, making
it more competitive with the hash and merge joins. For Q1, for every tuple in auth, a scan on
the publ.pubid index needs to be made; since a hashtable allows faster (constant vs logarithmic) scanning,
it pays off to create such a hashtable before the join.

For Q2, most tuples in auth don't need to be considered (wrong author name), this means very few
scans on the index on publ.pubid, so creating a hashtable does not pay off and an index nested loop
join is faster.

When also having an index on auth.pubid, a merge join becomes viable. Postgres however does not
remember whether or not the data is in the correct sort order (clustered), giving merge join a risk
of high cost.
\section{Nested Loop Join}

\paragraph{Response times}

\begin{flushleft}
\begin{tabular}{l|r|r}
  Indexes & Response time Q1 [ms] & Response time Q2 [ms] \\
  \hline
  index on {\tt Publ.pubID} & 73328.398 & 613.476  \\
  index on {\tt Auth.pubID} & 36287.789 & 31530.108 \\
  index on {\tt Publ.pubID} and {\tt Auth.pubID} & 35963.802 & 610.675 \\
\end{tabular}
\end{flushleft}

\paragraph{Query plans}\mbox{}\\ 

\noindent Index on {\tt Publ.pubID} (Q1/Q2):
{\small
\begin{verbatim}
Q1:
Nested Loop  (cost=0.43..3152174.22 rows=4158069 width=82)
(actual time=0.072..72166.924 rows=3095201 loops=1)
   ->  Seq Scan on auth  (cost=0.00..115496.01 rows=6188901 width=38)
       (actual time=0.032..1824.825 rows=3095201 loops=1)
   ->  Index Scan using publ_pubid_idx on publ  (cost=0.43..0.48 rows=1 width=89)
       (actual time=0.020..0.021 rows=1 loops=3095201)
         Index Cond: ((pubid)::text = (auth.pubid)::text)
 Planning time: 0.148 ms
 Execution time: 73328.398 ms

Q2:
Nested Loop  (cost=0.43..131374.22 rows=32 width=67)
(actual time=0.367..613.352 rows=183 loops=1)
   ->  Seq Scan on auth  (cost=0.00..130968.26 rows=48 width=23)
       (actual time=0.317..607.170 rows=183 loops=1)
         Filter: ((name)::text = 'Divesh Srivastava'::text)
         Rows Removed by Filter: 3095018
   ->  Index Scan using publ_pubid_idx on publ  (cost=0.43..8.45 rows=1 width=89)
       (actual time=0.030..0.030 rows=1 loops=183)
         Index Cond: ((pubid)::text = (auth.pubid)::text)
 Planning time: 0.145 ms
 Execution time: 613.476 ms
\end{verbatim}
}

\noindent Index on {\tt Auth.pubID} (Q1/Q2):
{\small
\begin{verbatim}
Q1:
Nested Loop  (cost=0.43..979609.42 rows=3095201 width=82)
(actual time=86.803..35148.185 rows=3095201 loops=1)
   ->  Seq Scan on publ  (cost=0.00..57126.14 rows=1233214 width=89)
       (actual time=86.717..746.490 rows=1233214 loops=1)
   ->  Index Scan using auth_pubid_idx on auth  (cost=0.43..0.72 rows=3 width=38)
       (actual time=0.023..0.025 rows=3 loops=1233214)
         Index Cond: ((pubid)::text = (publ.pubid)::text)
 Planning time: 0.237 ms
 Execution time: 36287.789 ms

Q2:
Nested Loop  (cost=0.43..833491.25 rows=412 width=67)
(actual time=2363.851..31529.980 rows=183 loops=1)
   ->  Seq Scan on publ  (cost=0.00..53815.96 rows=902196 width=89)
       (actual time=86.556..765.840 rows=1233214 loops=1)
   ->  Index Scan using auth_pubid_idx on auth  (cost=0.43..0.85 rows=1 width=23)
       (actual time=0.024..0.024 rows=0 loops=1233214)
         Index Cond: ((pubid)::text = (publ.pubid)::text)
         Filter: ((name)::text = 'Divesh Srivastava'::text)
         Rows Removed by Filter: 3
 Planning time: 0.297 ms
 Execution time: 31530.108 ms
\end{verbatim}
}

\noindent Index on {\tt Auth.pubID} and {\tt Auth.pubID} (Q1/Q2):
{\small
\begin{verbatim}
Q1:
Nested Loop  (cost=0.43..1053602.26 rows=3087776 width=82)
(actual time=85.901..34807.160 rows=3095201 loops=1)
   ->  Seq Scan on publ  (cost=0.00..57126.14 rows=1233214 width=89)
       (actual time=85.817..741.700 rows=1233214 loops=1)
   ->  Index Scan using auth_pubid_idx on auth  (cost=0.43..0.76 rows=5 width=38)
       (actual time=0.023..0.024 rows=3 loops=1233214)
         Index Cond: ((pubid)::text = (publ.pubid)::text)
 Planning time: 0.359 ms
 Execution time: 35963.802 ms

Q2:
Nested Loop  (cost=0.43..95636.69 rows=412 width=67)
(actual time=0.420..610.558 rows=183 loops=1)
   ->  Seq Scan on auth  (cost=0.00..92204.20 rows=412 width=23)
       (actual time=0.358..604.273 rows=183 loops=1)
         Filter: ((name)::text = 'Divesh Srivastava'::text)
         Rows Removed by Filter: 3095018
   ->  Index Scan using publ_pubid_idx on publ  (cost=0.43..8.32 rows=1 width=89)
       (actual time=0.030..0.031 rows=1 loops=183)
         Index Cond: ((pubid)::text = (auth.pubid)::text)
 Planning time: 0.166 ms
 Execution time: 610.675 ms
\end{verbatim}
}

\paragraph{Discussion}
An index on Publ.pubid gives a big speedup for Q2, since it allows
scanning Auth for the ca. 400 pubids with the correct author,
then looking up the extra data for the pubid using the index on Publ.pubid.
This way, all blocks of Auth need to be accessed once, plus a small fraction
of the blocks in Publ. When there is no index on Publ.pubid, for every
matching pubid, all of publ needs to be scanned for this pubid, so publ
needs to be scanned ca. 400 times.

For Q1, an index on Auth.pubid is somewhat better than one on Publ.pubid,
since Auth has about 3 times as many tuples. Publ can then be scanned in the
outer loop, the index on Auth queried in the inner. This is faster than the
other order because in the outer loop the cost grows linearly (number of iterations)
and in the inner logarithmically (index access cost).


\section{Sort-Merge Join}

\paragraph{Response times}

\begin{flushleft}
\begin{tabular}{l|r|r}
  Indexes & Response time Q1 [ms] & Response time Q2 [ms] \\
  \hline
  no index & 121287.21 & 27985.16 \\
  two non-clustering indexes & 30889.430 & 2848.947 \\
  two clustering indexes & 18913.409 & 2506.367  \\
\end{tabular}
\end{flushleft}

\paragraph{Query plans}\mbox{}\\ 

\noindent No index (Q1/Q2):
{\small
\begin{verbatim}
Q1:
Merge Join  (cost=846625.43..906942.51 rows=3095201 width=82)
(actual time=89743.785..120099.323 rows=3095201 loops=1)
   Merge Cond: ((publ.pubid)::text = (auth.pubid)::text)
   ->  Sort  (cost=285913.35..288996.38 rows=1233214 width=89)
       (actual time=23248.235..26457.879 rows=1233208 loops=1)
         Sort Key: publ.pubid
         Sort Method: external meMerge Join
         (cost=351402.53..357553.95 rows=24 width=67)
         (actual time=24166.614..27962.238 rows=183 loops=1)
   Merge Cond: ((publ.pubid)::text = (auth.pubid)::text)
   ->  Sort  (cost=285913.35..288996.38 rows=1233214 width=89)
       (actual time=23291.779..26285.255 rows=1229958 loops=1)
         Sort Key: publ.pubid
         Sort Method: external merge  Disk: 121400kB
         ->  Seq Scan on publ  (cost=0.00..34694.14 rows=1233214 width=89)
             (actual time=0.011..857.711 rows=1233214 loops=1)
   ->  Sort  (cost=65488.56..65488.62 rows=24 width=23)
       (actual time=526.541..526.632 rows=183 loops=1)
         Sort Key: auth.pubid
         Sort Method: quicksort  Memory: 39kB
         ->  Seq Scan on auth  (cost=0.00..65488.01 rows=24 width=23)
             (actual time=43.402..525.537 rows=183 loops=1)
               Filter: ((name)::text = 'Divesh Srivastava'::text)
               Rows Removed by Filter: 3095018
 Planning time: 0.347 ms
 Execution time: 27985.160 msrge  Disk: 121400kB
         ->  Seq Scan on publ  (cost=0.00..34694.14 rows=1233214 width=89)
             (actual time=0.006..872.339 rows=1233214 loops=1)
   ->  Materialize  (cost=560711.47..576187.47 rows=3095201 width=38)
       (actual time=66495.525..78988.422 rows=3095201 loops=1)
         ->  Sort  (cost=560711.47..568449.47 rows=3095201 width=38)
             (actual time=66495.518..76637.072 rows=3095201 loops=1)
               Sort Key: auth.pubid
               Sort Method: external merge  Disk: 148096kB
               ->  Seq Scan on auth  (cost=0.00..57750.01 rows=3095201 width=38)
                   (actual time=0.050..1802.142 rows=3095201 loops=1)
 Planning time: 0.342 ms
 Execution time: 121287.210 ms

Q2:
Merge Join  (cost=351402.53..357553.95 rows=24 width=67)
(actual time=24166.614..27962.238 rows=183 loops=1)
   Merge Cond: ((publ.pubid)::text = (auth.pubid)::text)
   ->  Sort  (cost=285913.35..288996.38 rows=1233214 width=89)
       (actual time=23291.779..26285.255 rows=1229958 loops=1)
         Sort Key: publ.pubid
         Sort Method: external merge  Disk: 121400kB
         ->  Seq Scan on publ  (cost=0.00..34694.14 rows=1233214 width=89)
             (actual time=0.011..857.711 rows=1233214 loops=1)
   ->  Sort  (cost=65488.56..65488.62 rows=24 width=23)
       (actual time=526.541..526.632 rows=183 loops=1)
         Sort Key: auth.pubid
         Sort Method: quicksort  Memory: 39kB
         ->  Seq Scan on auth  (cost=0.00..65488.01 rows=24 width=23)
             (actual time=43.402..525.537 rows=183 loops=1)
               Filter: ((name)::text = 'Divesh Srivastava'::text)
               Rows Removed by Filter: 3095018
 Planning time: 0.347 ms
 Execution time: 27985.160 ms
\end{verbatim}
}

\noindent Two non-clustering indexes (Q1/Q2):
{\small
\begin{verbatim}
Q1:
Merge Join  (cost=0.86..263662.50 rows=3095201 width=82) (actual time=0.021..29647.748 rows=3095201 loops=1)
   Merge Cond: ((publ.pubid)::text = (auth.pubid)::text)
   ->  Index Scan using publ_pubid_idx on publ  (cost=0.43..73268.61 rows=1233214 width=89) (actual time=0.005..1058.483 rows=1233208 loops=1)
   ->  Index Scan using auth_pubid_idx on auth  (cost=0.43..148799.09 rows=3095201 width=38) (actual time=0.004..12859.194 rows=3095201 loops=1)
 Planning time: 0.679 ms
 Execution time: 30889.430 ms

Q2:
Merge Join  (cost=92222.52..197223.72 rows=412 width=67)
(actual time=776.238..2848.822 rows=183 loops=1)
   Merge Cond: ((publ.pubid)::text = (auth.pubid)::text)
   ->  Index Scan using publ_pubid_idx on publ  (cost=0.43..101912.41 rows=1233214 width=89)
       (actual time=0.006..1050.215 rows=1229958 loops=1)
   ->  Sort  (cost=92222.09..92223.12 rows=412 width=23)
       (actual time=605.513..605.624 rows=183 loops=1)
         Sort Key: auth.pubid
         Sort Method: quicksort  Memory: 39kB
         ->  Seq Scan on auth  (cost=0.00..92204.20 rows=412 width=23)
             (actual time=0.301..604.428 rows=183 loops=1)
               Filter: ((name)::text = 'Divesh Srivastava'::text)
               Rows Removed by Filter: 3095018
 Planning time: 0.557 ms
 Execution time: 2848.947 ms
\end{verbatim}
}

\noindent Two clustering indexes  (Q1/Q2):
{\small
\begin{verbatim}
Q1:
Merge Join  (cost=0.86..263148.44 rows=3095201 width=82)
(actual time=0.036..17714.296 rows=3095201 loops=1)
   Merge Cond: ((publ.pubid)::text = (auth.pubid)::text)
   ->  Index Scan using publ_pubid_idx on publ  (cost=0.43..73286.56 rows=1233214 width=89)
       (actual time=0.011..880.471 rows=1233208 loops=1)
   ->  Index Scan using auth_pubid_idx on auth  (cost=0.43..148450.94 rows=3095201 width=38)
       (actual time=0.008..1985.179 rows=3095201 loops=1)
 Planning time: 0.594 ms
 Execution time: 18913.409 ms

Q2:
Merge Join  (cost=65518.38..141893.74 rows=413 width=67)
(actual time=671.547..2506.252 rows=183 loops=1)
   Merge Cond: ((publ.pubid)::text = (auth.pubid)::text)
   ->  Index Scan using publ_pubid_idx on publ  (cost=0.43..73286.56 rows=1233214 width=89)
       (actual time=0.005..834.890 rows=1229958 loops=1)
   ->  Sort  (cost=65517.96..65518.99 rows=413 width=23)
       (actual time=520.571..520.667 rows=183 loops=1)
         Sort Key: auth.pubid
         Sort Method: quicksort  Memory: 39kB
         ->  Seq Scan on auth  (cost=0.00..65500.01 rows=413 width=23)
             (actual time=45.314..520.250 rows=183 loops=1)
               Filter: ((name)::text = 'Divesh Srivastava'::text)
               Rows Removed by Filter: 3095018
 Planning time: 0.583 ms
 Execution time: 2506.367 ms
\end{verbatim}
}

\paragraph{Discussion}
Without an index it was clear, that it takes a while, since an external merge sort is needed. This takes a lot of read and write time.
The difference between the non-clustering indices and clustering indices wasn't so clear. We thought the clustering indices would have a bigger improvement, since they are already sorted. When looking at the execution plans, we saw that both Q1 queries are the same and don't use a sort at all. This explains the small difference. Because the tuples are sorted, we need a few less block accesses, and therefore the clustering is a little faster.
We think for Q2 the same explanation is true, since the execution plan is also the same.

\section{Hash Join}

\paragraph{Response times}

\begin{flushleft}
\begin{tabular}{l|r|r}
  Indexes & Response time Q1 [ms] & Response time [ms] Q2 \\
  \hline
  no index & 9473.583 & 1686.848 \\
\end{tabular}
\end{flushleft}

\paragraph{Query plans}\mbox{}\\ 

\noindent No Index (Q1/Q2):
{\small
\begin{verbatim}
Q1:
Hash Join  (cost=69713.26..257884.36 rows=3215760 width=82)
(actual time=1992.181..8310.008 rows=3095201 loops=1)
   Hash Cond: ((auth.pubid)::text = (publ.pubid)::text)
   ->  Seq Scan on auth  (cost=0.00..58955.60 rows=3215760 width=38)
       (actual time=0.019..1697.798 rows=3095201 loops=1)
   ->  Hash  (cost=35108.34..35108.34 rows=1274634 width=89)
       (actual time=1990.989..1990.989 rows=1233214 loops=1)
         Buckets: 4096  Batches: 64  Memory Usage: 2344kB
         ->  Seq Scan on publ  (cost=0.00..35108.34 rows=1274634 width=89)
             (actual time=0.059..951.548 rows=1233214 loops=1)
 Planning time: 0.652 ms
 Execution time: 9473.583 ms

Q2:
Hash Join  (cost=65488.31..140262.15 rows=24 width=67)
(actual time=629.617..1686.721 rows=183 loops=1)
   Hash Cond: ((publ.pubid)::text = (auth.pubid)::text)
   ->  Seq Scan on publ  (cost=0.00..34694.14 rows=1233214 width=89)
       (actual time=0.009..587.001 rows=1233214 loops=1)
   ->  Hash  (cost=65488.01..65488.01 rows=24 width=23)
       (actual time=539.756..539.756 rows=183 loops=1)
         Buckets: 1024  Batches: 1  Memory Usage: 11kB
         ->  Seq Scan on auth  (cost=0.00..65488.01 rows=24 width=23)
             (actual time=44.442..539.497 rows=183 loops=1)
               Filter: ((name)::text = 'Divesh Srivastava'::text)
               Rows Removed by Filter: 3095018
 Planning time: 0.111 ms
 Execution time: 1686.848 ms
\end{verbatim}
}

\paragraph{Discussion}

What do you think about the response time of the hash index vs.\ the
response times of sort-merge and index nested loop join for each of
the queries? Explain.

\bigskip

\noindent Time in hours per person: {\bf 3.5}

\bigskip

\begin{center}
  \begin{tabular}{c}
    \hline
    {\bf Important:} Reference your information sources!
    \\\hline
  \end{tabular}
\end{center}

\end{document}
