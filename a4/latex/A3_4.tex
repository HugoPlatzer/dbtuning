\documentclass[11pt]{scrartcl}

\usepackage[top=2cm]{geometry}
%\pagestyle{empty}

\title{
  \textbf{\large Database Tuning -- Assignment 4}\\
  Index Tuning
}

\author{
 Group Name (e.g. A1, B5, B3)\\
 \large Lastname1 Firstname1, StudentID1 \\
 \large Lastname2 Firstname2, StudentID2 \\
 \large Lastname3 Firstname3, StudentID3 
}

\begin{document}

\maketitle

\noindent
{\it Notes:}
\begin{itemize}\itemsep=0pt
\item Do not forget to run {\tt ANALYZE tablename} after creating or
  changing a table.
\item Use {\tt EXPLAIN ANALYZE} for the query plans that you display in the report.
\end{itemize}


\section{Experimental Setup}

How do send the queries to the database? How do you measure the
execution time for a sequence of queries?

\section{Clustered B$^+$-Tree Index}

\paragraph{Point Query}

Repeat the following query multiple times with different conditions for {\tt pubID}.

{\small
\begin{verbatim}
SELECT * FROM Publ WHERE pubID = ...
\end{verbatim}
}

\noindent
Which conditions did you use?

\smallskip\noindent
Show the runtime results and compute the throughput.

\smallskip\noindent
Query plan (for one of the queries):
{\small
\begin{verbatim}
 Index Scan using publ_idx on publ  (cost=0.43..8.45 rows=1 width=112)
 (actual time=0.026..0.026 rows=0 loops=1)
   Index Cond: ((pubid)::text = ' books/acm/Kim95'::text)
 Planning time: 0.153 ms
 Execution time: 0.050 ms
\end{verbatim}
}

\paragraph{Multipoint Query -- Low Selectivity}

Repeat the following query multiple times with different conditions for {\tt booktitle}.

{\small
\begin{verbatim}
SELECT * FROM Publ WHERE booktitle = ...
\end{verbatim}
}

\noindent
Which conditions did you use?

\smallskip\noindent
Show the runtime results and compute the throughput.

\smallskip\noindent
Query plan (for one of the queries):
{\small
\begin{verbatim}
 Index Scan using publ_idx on publ  (cost=0.43..569.73 rows=179 width=112)
 (actual time=0.040..0.089 rows=97 loops=1)
   Index Cond: ((booktitle)::text = 'Z User Workshop'::text)
 Planning time: 0.106 ms
 Execution time: 0.150 ms
\end{verbatim}
}


\paragraph{Multipoint Query -- High Selectivity}

Repeat the following query multiple times with different conditions for {\tt year}.

{\small
\begin{verbatim}
SELECT * FROM Publ WHERE year = ...
\end{verbatim}
}

\noindent
Which conditions did you use?

\smallskip\noindent
Show the runtime results and compute the throughput.

\smallskip\noindent
Query plan (for one of the queries):
{\small
\begin{verbatim}
query plan
\end{verbatim}
}

\section{Non-Clustered B$^+$-Tree Index}

\noindent \emph{Note:} Make sure the data is not physically ordered by
the indexed attributes due to the clustering index that you created
before.

\paragraph{Point Query}

Repeat the following query multiple times with different conditions for {\tt pubID}.

{\small
\begin{verbatim}
SELECT * FROM Publ WHERE pubID = ...
\end{verbatim}
}

\noindent
Which conditions did you use?

\smallskip\noindent
Show the runtime results and compute the throughput.

\smallskip\noindent
Query plan (for one of the queries):
{\small
\begin{verbatim}
 Index Scan using publ_idx on publ  (cost=0.43..8.45 rows=1 width=112)
 (actual time=0.026..0.026 rows=0 loops=1)
   Index Cond: ((pubid)::text = ' books/acm/Kim95'::text)
 Planning time: 0.154 ms
 Execution time: 0.049 ms
\end{verbatim}
}


\paragraph{Multipoint Query -- Low Selectivity}

Repeat the following query multiple times with different conditions for {\tt booktitle}.

{\small
\begin{verbatim}
SELECT * FROM Publ WHERE booktitle = ...
\end{verbatim}
}

\noindent
Which conditions did you use?

\smallskip\noindent
Show the runtime results and compute the throughput.

\smallskip\noindent
Query plan (for one of the queries):
{\small
\begin{verbatim}
 Index Scan using publ_idx on publ  (cost=0.43..569.73 rows=179 width=112) (actual time=0.050..0.119 rows=97 loops=1)
   Index Cond: ((booktitle)::text = 'Z User Workshop'::text)
 Planning time: 0.275 ms
 Execution time: 0.178 ms
\end{verbatim}
}


\paragraph{Multipoint Query -- High Selectivity}

Repeat the following query multiple times with different conditions for {\tt year}.

{\small
\begin{verbatim}
SELECT * FROM Publ WHERE year = ...
\end{verbatim}
}

\noindent
Which conditions did you use?

\smallskip\noindent
Show the runtime results and compute the throughput.

\smallskip\noindent
Query plan (for one of the queries):
{\small
\begin{verbatim}
query plan
\end{verbatim}
}

\section{Non-Clustered Hash Index}

\noindent \emph{Note:} Make sure the data is not physically ordered by
the indexed attributes due to the clustering index that you created
before.

\paragraph{Point Query}

Repeat the following query multiple times with different conditions for {\tt pubID}.

{\small
\begin{verbatim}
SELECT * FROM Publ WHERE pubID = ...
\end{verbatim}
}

\noindent
Which conditions did you use?

\smallskip\noindent
Show the runtime results and compute the throughput.

\smallskip\noindent
Query plan (for one of the queries):
{\small
\begin{verbatim}
 Index Scan using publ_idx on publ  (cost=0.00..8.02 rows=1 width=112)
 (actual time=0.020..0.020 rows=0 loops=1)
   Index Cond: ((pubid)::text = ' books/acm/Kim95'::text)
 Planning time: 0.210 ms
 Execution time: 0.048 ms
\end{verbatim}
}


\paragraph{Multipoint Query -- Low Selectivity}

Repeat the following query multiple times with different conditions for {\tt booktitle}.

{\small
\begin{verbatim}
SELECT * FROM Publ WHERE booktitle = ...
\end{verbatim}
}

\noindent
Which conditions did you use?

\smallskip\noindent
Show the runtime results and compute the throughput.

\smallskip\noindent
Query plan (for one of the queries):
{\small
\begin{verbatim}
query plan
\end{verbatim}
}


\paragraph{Multipoint Query -- High Selectivity}

Repeat the following query multiple times with different conditions for {\tt year}.

{\small
\begin{verbatim}
SELECT * FROM Publ WHERE year = ...
\end{verbatim}
}

\noindent
Which conditions did you use?

\smallskip\noindent
Show the runtime results and compute the throughput.

\smallskip\noindent
Query plan (for one of the queries):
{\small
\begin{verbatim}
query plan
\end{verbatim}
}


\section{Table Scan}

\noindent \emph{Note:} Make sure the data is not physically ordered by
the indexed attributes due to the clustering index that you created
before.

\paragraph{Point Query}

Repeat the following query multiple times with different conditions for {\tt pubID}.

{\small
\begin{verbatim}
SELECT * FROM Publ WHERE pubID = ...
\end{verbatim}
}

\noindent
Which conditions did you use?

\smallskip\noindent
Show the runtime results and compute the throughput.

\smallskip\noindent
Query plan (for one of the queries):
{\small
\begin{verbatim}
 Seq Scan on publ  (cost=0.00..37841.18 rows=1 width=112)
 (actual time=359.876..359.876 rows=0 loops=1)
   Filter: ((pubid)::text = ' books/acm/Kim95'::text)
   Rows Removed by Filter: 1233214
 Planning time: 0.108 ms
 Execution time: 359.903 ms
\end{verbatim}
}


\paragraph{Multipoint Query -- Low Selectivity}

Repeat the following query multiple times with different conditions for {\tt booktitle}.

{\small
\begin{verbatim}
SELECT * FROM Publ WHERE booktitle = ...
\end{verbatim}
}

\noindent
Which conditions did you use?

\smallskip\noindent
Show the runtime results and compute the throughput.

\smallskip\noindent
Query plan (for one of the queries):
{\small
\begin{verbatim}
 Seq Scan on publ  (cost=0.00..37843.18 rows=179 width=112)
 (actual time=172.752..276.457 rows=97 loops=1)
   Filter: ((booktitle)::text = 'Z User Workshop'::text)
   Rows Removed by Filter: 1233117
 Planning time: 0.143 ms
 Execution time: 276.524 ms
\end{verbatim}
}


\paragraph{Multipoint Query -- High Selectivity}

Repeat the following query multiple times with different conditions for {\tt year}.

{\small
\begin{verbatim}
SELECT * FROM Publ WHERE year = ...
\end{verbatim}
}

\noindent
Which conditions did you use?

\smallskip\noindent
Show the runtime results and compute the throughput.

\smallskip\noindent
Query plan (for one of the queries):
{\small
\begin{verbatim}
query plan
\end{verbatim}
}

\section{Discussion}

Give the throughput of the query types and index types in queries/second.
\begin{center}
  \begin{tabular}{c|c|c|c|c}
    & clustered & non-clust.\ B$^+$-tree & non-clust.\ hash & table scan \\
    \hline
    point ({\tt pubID}) & & & \\
    \hline
    multipoint ({\tt booktitle}) & & & \\
    \hline
    multipoint  ({\tt year}) & & & \\  
  \end{tabular}
\end{center}

\medskip

Discuss the runtime results for the different index types and the
table scan. Are the results expected? Why / why not?


\bigskip

\noindent Time in hours per person: {\bf XXX}

\bigskip

\begin{center}
  \begin{tabular}{c}
    \hline
    {\bf Important:} Reference your information sources!
    \\\hline
  \end{tabular}
\end{center}

\end{document}
\grid
