\documentclass[11pt]{scrartcl}

\usepackage[top=2cm]{geometry}
%\pagestyle{empty}

\title{
  \textbf{\large Database Tuning -- Assignment 4}\\
  Index Tuning
}

\author{
 Group Name A3\\
 \large Platzer Hugo, 1421579 \\
 \large Strohmeier Mario, 1422959 \\
}

\begin{document}

\maketitle

\noindent
{\it Notes:}
\begin{itemize}\itemsep=0pt
\item Do not forget to run {\tt ANALYZE tablename} after creating or
  changing a table.
\item Use {\tt EXPLAIN ANALYZE} for the query plans that you display in the report.
\end{itemize}


\section{Experimental Setup}

How do send the queries to the database? How do you measure the
execution time for a sequence of queries?\\
We send the Queries from a Java program. The time was measured with System.nanoTime() from Java.

\section{Clustered B$^+$-Tree Index}

\paragraph{Point Query}

Repeat the following query multiple times with different conditions for {\tt pubID}.

{\small
\begin{verbatim}
SELECT * FROM Publ WHERE pubID = ...
\end{verbatim}
}

\noindent
\newcommand{\condA}[1][20000]{
One query was executed for each of the #1 lexicographically smallest pubIDs,
in ascending order.
}
\condA

\smallskip\noindent
Runtime: 64.1s $\Rightarrow$ 0.0032s/q\\
Throughput: 312q/s\\

\smallskip\noindent
Query plan (for one of the queries):
{\small
\begin{verbatim}
 Index Scan using publ_idx on publ  (cost=0.43..8.45 rows=1 width=112)
 (actual time=0.026..0.026 rows=0 loops=1)
   Index Cond: ((pubid)::text = ' books/acm/Kim95'::text)
 Planning time: 0.153 ms
 Execution time: 0.050 ms
\end{verbatim}
}

\paragraph{Multipoint Query -- Low Selectivity}

Repeat the following query multiple times with different conditions for {\tt booktitle}.

{\small
\begin{verbatim}
SELECT * FROM Publ WHERE booktitle = ...
\end{verbatim}
}

\noindent
\newcommand{\condB}[1][4841]{
One query was executed for the lexicographically smallest #1 booktitles in the dataset,
in ascending order.}
\condB

\smallskip\noindent
Runtime: 20.8s $\Rightarrow$ 0.0043s/q\\
Throughput: 233q/s\\

\smallskip\noindent
Query plan (for one of the queries):
{\small
\begin{verbatim}
 Index Scan using publ_idx on publ  (cost=0.43..569.73 rows=179 width=112)
 (actual time=0.040..0.089 rows=97 loops=1)
   Index Cond: ((booktitle)::text = 'Z User Workshop'::text)
 Planning time: 0.106 ms
 Execution time: 0.150 ms
\end{verbatim}
}


\paragraph{Multipoint Query -- High Selectivity}

Repeat the following query multiple times with different conditions for {\tt year}.

{\small
\begin{verbatim}
SELECT * FROM Publ WHERE year = ...
\end{verbatim}
}

\noindent
\newcommand{\condC}{
One query was executed for all 74 years in the dataset, in ascending order.
}
\condC

\smallskip\noindent
Runtime: 7.8s $\Rightarrow$ 0.105s/q\\
Throughput: 9.5q/s\\

\smallskip\noindent
Query plan (for one of the queries):
{\small
\begin{verbatim}
 Bitmap Heap Scan on publ  (cost=46.38..6818.57 rows=2317 width=112)
 (actual time=0.052..0.057 rows=12 loops=1)
   Recheck Cond: ((year)::text = '1936'::text)
   Heap Blocks: exact=1
   ->  Bitmap Index Scan on publ_idx  (cost=0.00..45.80 rows=2317 width=0)
   (actual time=0.041..0.041 rows=12 loops=1)
         Index Cond: ((year)::text = '1936'::text)
 Planning time: 0.239 ms
 Execution time: 0.096 ms
\end{verbatim}
}

\section{Non-Clustered B$^+$-Tree Index}

\noindent \emph{Note:} Make sure the data is not physically ordered by
the indexed attributes due to the clustering index that you created
before.

\paragraph{Point Query}

Repeat the following query multiple times with different conditions for {\tt pubID}.

{\small
\begin{verbatim}
SELECT * FROM Publ WHERE pubID = ...
\end{verbatim}
}

\noindent
Which conditions did you use?\\
\condA

\smallskip\noindent
Runtime: 63.9s $\Rightarrow$ 0.0032s/q\\
Throughput: 313q/s\\

\smallskip\noindent
Query plan (for one of the queries):
{\small
\begin{verbatim}
 Index Scan using publ_idx on publ  (cost=0.43..8.45 rows=1 width=112)
 (actual time=0.026..0.026 rows=0 loops=1)
   Index Cond: ((pubid)::text = ' books/acm/Kim95'::text)
 Planning time: 0.154 ms
 Execution time: 0.049 ms
\end{verbatim}
}


\paragraph{Multipoint Query -- Low Selectivity}

Repeat the following query multiple times with different conditions for {\tt booktitle}.

{\small
\begin{verbatim}
SELECT * FROM Publ WHERE booktitle = ...
\end{verbatim}
}

\noindent
\condB

\smallskip\noindent
Show the runtime results and compute the throughput.\\
Runtime: 22.1s $\Rightarrow$ 0.0046s/q\\
Throughput: 219q/s\\

\smallskip\noindent
Query plan (for one of the queries):
{\small
\begin{verbatim}
 Index Scan using publ_idx on publ  (cost=0.43..569.73 rows=179 width=112)
  (actual time=0.050..0.119 rows=97 loops=1)
   Index Cond: ((booktitle)::text = 'Z User Workshop'::text)
 Planning time: 0.275 ms
 Execution time: 0.178 ms
\end{verbatim}
}


\paragraph{Multipoint Query -- High Selectivity}

Repeat the following query multiple times with different conditions for {\tt year}.

{\small
\begin{verbatim}
SELECT * FROM Publ WHERE year = ...
\end{verbatim}
}

\noindent
Which conditions did you use?\\
\condC

\smallskip\noindent
Show the runtime results and compute the throughput.\\
Runtime: 7.9s $\Rightarrow$ 0.106s/q\\
Throughput: 9.37q/s\\

\smallskip\noindent
Query plan (for one of the queries):
{\small
\begin{verbatim}
 Bitmap Heap Scan on publ  (cost=46.38..6818.52 rows=2317 width=112)
 (actual time=0.049..0.088 rows=12 loops=1)
   Recheck Cond: ((year)::text = '1936'::text)
   Heap Blocks: exact=7
   ->  Bitmap Index Scan on publ_idx  (cost=0.00..45.80 rows=2317 width=0)
   (actual time=0.038..0.038 rows=12 loops=1)
         Index Cond: ((year)::text = '1936'::text)
 Planning time: 0.255 ms
 Execution time: 0.123 ms

\end{verbatim}
}

\section{Non-Clustered Hash Index}

\noindent \emph{Note:} Make sure the data is not physically ordered by
the indexed attributes due to the clustering index that you created
before.

\paragraph{Point Query}

Repeat the following query multiple times with different conditions for {\tt pubID}.

{\small
\begin{verbatim}
SELECT * FROM Publ WHERE pubID = ...
\end{verbatim}
}

\noindent
\condA

\smallskip\noindent
Show the runtime results and compute the throughput.\\
Runtime: 65.8s $\Rightarrow$ 0.0033s/q\\
Throughput: 304q/s\\

\smallskip\noindent
Query plan (for one of the queries):
{\small
\begin{verbatim}
 Index Scan using publ_idx on publ  (cost=0.00..8.02 rows=1 width=112)
 (actual time=0.020..0.020 rows=0 loops=1)
   Index Cond: ((pubid)::text = ' books/acm/Kim95'::text)
 Planning time: 0.210 ms
 Execution time: 0.048 ms
\end{verbatim}
}


\paragraph{Multipoint Query -- Low Selectivity}

Repeat the following query multiple times with different conditions for {\tt booktitle}.

{\small
\begin{verbatim}
SELECT * FROM Publ WHERE booktitle = ...
\end{verbatim}
}

\noindent
\condB

\smallskip\noindent
Show the runtime results and compute the throughput.\\
Runtime: 21.6s $\Rightarrow$ 0.0045s/q\\
Throughput: 224q/s\\

\smallskip\noindent
Query plan (for one of the queries):
{\small
\begin{verbatim}
 Bitmap Heap Scan on publ  (cost=5.39..675.65 rows=179 width=112)
 (actual time=0.043..0.128 rows=97 loops=1)
   Recheck Cond: ((booktitle)::text = 'Z User Workshop'::text)
   Heap Blocks: exact=6
   ->  Bitmap Index Scan on publ_idx  (cost=0.00..5.34 rows=179 width=0)
   (actual time=0.025..0.025 rows=97 loops=1)
         Index Cond: ((booktitle)::text = 'Z User Workshop'::text)
 Planning time: 0.249 ms
 Execution time: 0.197 ms
\end{verbatim}
}


\paragraph{Multipoint Query -- High Selectivity}

Repeat the following query multiple times with different conditions for {\tt year}.

{\small
\begin{verbatim}
SELECT * FROM Publ WHERE year = ...
\end{verbatim}
}

\noindent
\condC

\smallskip\noindent
Show the runtime results and compute the throughput.\\
Runtime: 7s $\Rightarrow$ 0.095s/q\\
Throughput: 10.6q/s\\

\smallskip\noindent
Query plan (for one of the queries):
{\small
\begin{verbatim}
 Bitmap Heap Scan on publ  (cost=73.96..6846.09 rows=2317 width=112)
 (actual time=0.034..0.076 rows=12 loops=1)
   Recheck Cond: ((year)::text = '1936'::text)
   Heap Blocks: exact=7
   ->  Bitmap Index Scan on publ_idx  (cost=0.00..73.38 rows=2317 width=0)
   (actual time=0.016..0.016 rows=12 loops=1)
         Index Cond: ((year)::text = '1936'::text)
 Planning time: 0.200 ms
 Execution time: 0.113 ms
\end{verbatim}
}


\section{Table Scan}

\noindent \emph{Note:} Make sure the data is not physically ordered by
the indexed attributes due to the clustering index that you created
before.

\paragraph{Point Query}

Repeat the following query multiple times with different conditions for {\tt pubID}.

{\small
\begin{verbatim}
SELECT * FROM Publ WHERE pubID = ...
\end{verbatim}
}

\noindent
Which conditions did you use?\\
\condA[500]

\smallskip\noindent
Show the runtime results and compute the throughput.\\
Runtime: 125.9s $\Rightarrow$ 0.251s/q\\
Throughput: 4q/s\\

\smallskip\noindent
Query plan (for one of the queries):
{\small
\begin{verbatim}
 Seq Scan on publ  (cost=0.00..37841.18 rows=1 width=112)
 (actual time=359.876..359.876 rows=0 loops=1)
   Filter: ((pubid)::text = ' books/acm/Kim95'::text)
   Rows Removed by Filter: 1233214
 Planning time: 0.108 ms
 Execution time: 359.903 ms
\end{verbatim}
}


\paragraph{Multipoint Query -- Low Selectivity}

Repeat the following query multiple times with different conditions for {\tt booktitle}.

{\small
\begin{verbatim}
SELECT * FROM Publ WHERE booktitle = ...
\end{verbatim}
}

\noindent
\condB[300]

\smallskip\noindent
Show the runtime results and compute the throughput.\\
Runtime: 84.3s $\Rightarrow$ 0.281s/q\\
Throughput: 3.56q/s\\

\smallskip\noindent
Query plan (for one of the queries):
{\small
\begin{verbatim}
 Seq Scan on publ  (cost=0.00..37843.18 rows=179 width=112)
 (actual time=172.752..276.457 rows=97 loops=1)
   Filter: ((booktitle)::text = 'Z User Workshop'::text)
   Rows Removed by Filter: 1233117
 Planning time: 0.143 ms
 Execution time: 276.524 ms
\end{verbatim}
}


\paragraph{Multipoint Query -- High Selectivity}

Repeat the following query multiple times with different conditions for {\tt year}.

{\small
\begin{verbatim}
SELECT * FROM Publ WHERE year = ...
\end{verbatim}
}

\noindent
\condC

\smallskip\noindent
Show the runtime results and compute the throughput.\\
Runtime: 26.7s $\Rightarrow$ 0.361s/q\\
Throughput: 2.77q/s\\


\smallskip\noindent
Query plan (for one of the queries):
{\small
\begin{verbatim}
 Seq Scan on publ  (cost=0.00..37841.18 rows=2317 width=112)
 (actual time=75.715..299.065 rows=12 loops=1)
   Filter: ((year)::text = '1936'::text)
   Rows Removed by Filter: 1233202
 Planning time: 0.068 ms
 Execution time: 299.103 ms
\end{verbatim}
}

\section{Discussion}

Give the throughput of the query types and index types in queries/second.
\begin{center}
  \begin{tabular}{c|c|c|c|c}
    & clustered & non-clust.\ B$^+$-tree & non-clust.\ hash & table scan \\
    \hline
    point ({\tt pubID}) & 312 & 313 & 304 & 4\\
    \hline
    multipoint ({\tt booktitle}) & 233 & 219 & 224 & 3.56 \\
    \hline
    multipoint  ({\tt year}) & 9.5 & 9.37 & 10.6 & 2.77 \\  
  \end{tabular}
\end{center}

\medskip

The results were mostly as expected.\\
Indices on point queries have a significant performance improvement over a table scan, since only one tuple is returned.\\
For the multipoint query using the booktitle, the results were also expected, because only a few tuples are returned by the query. So it is more efficient to use the index.\\
The multipoint query using the year, however returns a lot of tuples. So the use of a table scan is not that much slower, compared to using the index.\\
The only result that was not expected is, that the clustering of the index, does not grant a performance improvement.

\bigskip

\noindent Time in hours per person: {\bf 4}

\bigskip

\begin{center}
  \begin{tabular}{c}
    \hline
    {\bf Important:} Reference your information sources!
    \\\hline
  \end{tabular}
\end{center}

\end{document}
\grid
