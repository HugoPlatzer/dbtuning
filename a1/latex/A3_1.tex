\documentclass[11pt]{scrartcl}

\usepackage{url,float}

\title{
  \textbf{\large Database Tuning -- Assignment 1}\\
  Uploading Data to the Database
}

\author{
 A3\\
 \large Platzer Hugo, 1421579 \\
 \large Strohmeier Mario, 1422959 \\
}

\begin{document}

\maketitle

\subsection*{Straightforward Implementation}

  \paragraph{Implementation}

  Describe in a few lines how this approach works and show the query
  that you use. You may also show small (!) code snippets if you think
  they help the understanding.

{\small
\begin{verbatim}
    MY SQL QUERY ...
\end{verbatim}
}

  \subsection*{Efficient Approach 1: (NAME)}

  \paragraph{Implementation}

  Describe in a few lines how this approach works and show the query
  that you use. You may also show small (!) code snippets if you think
  they help the understanding.


{\small
\begin{verbatim}
    MY SQL QUERY ...
\end{verbatim}
}

  \paragraph{Why is this approach efficient?}

  Explain, why this approach is more efficient than the
  straightforward approach. Where does the system save the time? Be
  clear and precise!
  
  Important: Cite the references that you used to answer this
  question, for example, with footnotes\footnote{PostgreSQL 9.0
    Documentation, Chapter 3.5,
    \url{http://www.postgresql.org/docs/9.0/static/tutorial-window.html}}.

  \paragraph{Tuning principle}

  Which tuning principle did you apply? Pick the one that describes
  this approach best (``thing globally, fix locally'' is too general).

    \subsection*{Efficient Approach 2: (NAME)}

  \paragraph{Implementation}

  Describe in a few lines how this approach works and show the query
  that you use. You may also show small (!) code snippets if you think
  they help the understanding.

{\small
\begin{verbatim}
    MY SQL QUERY ...
\end{verbatim}
}

  \paragraph{Why is this approach efficient?}

  Explain, why this approach is more efficient than the
  straightforward approach. Where does the system save the time? Be
  clear and precise!
  
  Important: Cite the references that you used to answer this
  question, for example, with footnotes\footnote{PostgreSQL 9.0
    Documentation, Chapter 3.5,
    \url{http://www.postgresql.org/docs/9.0/static/tutorial-window.html}}.

  \paragraph{Tuning principle}

  Which tuning principle did you apply? Pick the one that describes
  this approach best (``thing globally, fix locally'' is too general).

  \subsection*{Runtime Experiment}

  \begin{table}[H]
  \begin{tabular}{l|r}
    Approach & Runtime [sec] \\
    \hline
    Straightforward & ... \\
    Approach 1 (NAME) & ... \\
    Approach 2 (NAME) & ...     
  \end{tabular}
  \end{table}

  \bigskip

  \noindent Notes:
  \begin{itemize}
  \item For the straightforward approach you are allowed to import
    only a subset of the tuples (e.g., 10000 tuples) and estimate the
    overall runtime. The timings for all other approaches should be
    real measurements over the whole dataset.
  \item Specify the setting of the experiment, i.e., where is the
    database server (local machine, database server at the
    department), where is the client (wired/wireless network of the
    department)?
\end{itemize}

  \subsection*{Time Spent on this Assignment}

  Time in hours per person: {\bf XXX}

\end{document}
