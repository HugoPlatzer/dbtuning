\documentclass[11pt]{scrartcl}

\usepackage{url,float}

\title{
  \textbf{\large Database Tuning -- Assignment 1}\\
  Uploading Data to the Database
}

\author{
 A3\\
 \large Platzer Hugo, 1421579 \\
 \large Strohmeier Mario, 1422959 \\
}

\begin{document}

\maketitle

\subsection*{Straightforward Implementation}

  \paragraph{Implementation}

  Every tuple gets inserted individually, by iterating over each line from the tsv file and executing the insert.

{\small
\begin{verbatim}
    "INSERT INTO auth (name, pubid) values (?, ?)"
\end{verbatim}
}

  \subsection*{Efficient Approach 1: (Batch/Bulk Insert)}

  \paragraph{Implementation}

  With this method, we don't execute the insert for each tuple seperatly. We keep adding tuples to a batch, until it is big enough or there are no tuples left. After the batch is big enough, we execute the insert for our whole batch.


{\small
\begin{verbatim}
    "INSERT INTO auth (name, pubid) values (?, ?)"
\end{verbatim}
}

  \paragraph{Why is this approach efficient?}

  This approach is more efficent, since we don't need to write a log for every single insert. This frees a lot of disk access time. Furthermore there is a lot less connection overhead, because the packets we send are bigger.\footnote{https://dbresearch.uni-salzburg.at/teaching/2016ws/dbt/dbt_01-handout-1x1.pdf, Page 23-27}

  \paragraph{Tuning principle}

  start-up costs are high; running costs are low

    \subsection*{Efficient Approach 2: (NAME)}

  \paragraph{Implementation}

  Describe in a few lines how this approach works and show the query
  that you use. You may also show small (!) code snippets if you think
  they help the understanding.

{\small
\begin{verbatim}
    MY SQL QUERY ...
\end{verbatim}
}

  \paragraph{Why is this approach efficient?}

  Explain, why this approach is more efficient than the
  straightforward approach. Where does the system save the time? Be
  clear and precise!
  
  Important: Cite the references that you used to answer this
  question, for example, with footnotes\footnote{PostgreSQL 9.0
    Documentation, Chapter 3.5,
    \url{http://www.postgresql.org/docs/9.0/static/tutorial-window.html}}.

  \paragraph{Tuning principle}

  Which tuning principle did you apply? Pick the one that describes
  this approach best (``thing globally, fix locally'' is too general).

  \subsection*{Runtime Experiment}

  \begin{table}[H]
  \begin{tabular}{l|r}
    Approach & Runtime [sec] \\
    \hline
    Straightforward & ... \\
    Approach 1 (NAME) & ... \\
    Approach 2 (NAME) & ...     
  \end{tabular}
  \end{table}

  \bigskip

  \noindent Notes:
  \begin{itemize}
  \item For the straightforward approach you are allowed to import
    only a subset of the tuples (e.g., 10000 tuples) and estimate the
    overall runtime. The timings for all other approaches should be
    real measurements over the whole dataset.
  \item Specify the setting of the experiment, i.e., where is the
    database server (local machine, database server at the
    department), where is the client (wired/wireless network of the
    department)?
\end{itemize}

  \subsection*{Time Spent on this Assignment}

  Time in hours per person: {\bf XXX}

\end{document}
